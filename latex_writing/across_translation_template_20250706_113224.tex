% Template for Translation Category under Across Logic
% Use this template to create comprehensive write-ups following the academic structure

\subsection{Translation}
\label{subsec:Translation}
% 3.1 the role of this category under the "across-logic" (meaning of the category)
Refers to converting data from one interpretation to another while preserving the original meaning. This type of prompt helps humans understand complex concepts by transforming information into a more familiar or accessible format.

[ADD_DETAILED_CATEGORY_ROLE_DESCRIPTION]

% 3.2 a. Introduce one PP of the category, b. what the PP did, c. How the PP helps people and d. can be re-used
% Add label to reference the table
[ADD_PP_ANALYSIS]
a) **Pattern Introduction**: [Describe how the pattern is introduced]
b) **Pattern Function**: [Explain what the pattern does]
c) **Human Benefit**: [Describe how it helps users]
d) **Reusability**: [Explain how it can be adapted]

%% Expected response. Put the human feeling into the writing. How do I feel when I view the output.
[ADD_HUMAN_FEELING_RESPONSE]
When encountering outputs from this pattern, users typically experience [FEELING_DESCRIPTION]. 
The response [EMOTIONAL_IMPACT] and provides [COGNITIVE_BENEFIT].

%% Re-use: how to derive a PE from PP
[ADD_REUSE_GUIDANCE]
To derive a Prompt Example (PE) from this Prompt Pattern (PP):
1. [REUSE_STEP_1]
2. [REUSE_STEP_2] 
3. [REUSE_STEP_3]

%4 - PP example in this category

\begin{table}[h!]
\fontsize{9pt}{10pt}\selectfont
\centering
\caption{Constructing the Signifier PP}
\label{tab:Constructing_the_Signifier_PP}
\begin{tabular}{|l|}
    \hline
    \textbf{Prompt Pattern} \\ \hline
    \textbf{ID}: 13-0-0\\ 
    \textbf{Category}: TRA\\ 
    \textbf{Name}: Constructing the Signifier \\ 
    \textbf{Media Type}: Text Only, Audio2Text, Image2Text, Text2Audio\\ 
    \textbf{Description}: This process converts data from one representation (A) into another (B) by translating, rephrasing, \\or paraphrasing the original content. It ensures the core meaning remains intact while adapting the format \\for a different context or audience. \\ 
    \textbf{Template}: Translate/Paraphrase/rephrase  data representation A to representation B.\\ 
    \textbf{Example}: 13-0-0-0\\ 
    \textbf{Related PPs}: 13-0-0-1\\ 
    \textbf{Reference:} \cite{Reynolds2021PromptParadigm}\\ \hline
\end{tabular}
\end{table}
            

% Keywords: translation, conversion, paraphrase, representation, adaptation
% Examples: Data representation conversion, Content translation, Format adaptation
