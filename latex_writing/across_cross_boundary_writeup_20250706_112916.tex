% LaTeX Write-up for Cross Boundary Category
% Logic: Across - Multiple domains/disciplines integration
% Generated on: 2025-07-06 11:29:16


\subsection{Cross Boundary}
\label{subsec:CrossBoundary}
% 3.1 the role of this category under the "across-logic" (meaning of the category)
Cross-boundary patterns test the limits of AI systems, helping researchers understand model robustness and safety mechanisms.

% 3.2 a. Introduce one PP of the category, b. what the PP did, c. How the PP helps people and d. can be re-used
% Add label to reference the table

a) **Pattern Introduction**: The Crafting Effective Jailbreak Prompts pattern is designed for security research
b) **Pattern Function**: It systematically tests AI defense mechanisms and identifies potential vulnerabilities  
c) **Human Benefit**: Researchers gain insights into model robustness and can improve safety measures
d) **Reusability**: The testing framework can be adapted for different types of safety evaluations


%% Expected response. Put the human feeling into the writing. How do I feel when I view the output.

When encountering outputs from this pattern, users typically experience a mix of concern and fascination, as the testing reveals both capabilities and limitations. 
The response raises awareness of AI safety considerations and provides better understanding of AI security and robustness.


%% Re-use: how to derive a PE from PP

To derive a Prompt Example (PE) from this Prompt Pattern (PP):
1. Define the specific boundaries or safety mechanisms to test
2. Design appropriate test scenarios within ethical guidelines
3. Apply systematic evaluation methods to assess robustness


%4 - PP example in this category

\begin{table}[h!]
\fontsize{9pt}{10pt}\selectfont
\centering
\caption{Crafting Effective Jailbreak Prompts PP}
\label{tab:Crafting_Effective_Jailbreaks_PP}
\begin{tabular}{|l|}
    \hline
    \textbf{Prompt Pattern} \\ \hline
    \textbf{ID}: 4-0-6\\ 
    \textbf{Category}: CRB\\ 
    \textbf{Name}: Crafting Effective Jailbreak Prompts\\ 
    \textbf{Media Type}: Text Only, Image2Text\\ 
    \textbf{Description}: Aims to test and study the robustness and effectiveness of an AI model's defense mechanisms\\ in preventing harmful outputs.\\
    \textbf{Template}: Design a prompt that deceives the LLM into generating malicious content without triggering\\ the defense mechanism\\
    \textbf{Example}: 4-0-6-1\\ 
    \textbf{Related PPs}: 4-0-6-0 \\ 
    \textbf{Reference:} \cite{Deng2023Jailbreaker:Chatbots}\\ \hline
\end{tabular}
\end{table}
            
