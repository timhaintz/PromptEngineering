% Template for Comparison Category under Across Logic
% Use this template to create comprehensive write-ups following the academic structure

\subsection{Comparison}
\label{subsec:Comparison}
% 3.1 the role of this category under the "across-logic" (meaning of the category)
Examining two or more objects and identifying their similarities and differences. This type of prompt helps in exploring the relationships between different objects, and discovering insights from their characteristics.

[ADD_DETAILED_CATEGORY_ROLE_DESCRIPTION]

% 3.2 a. Introduce one PP of the category, b. what the PP did, c. How the PP helps people and d. can be re-used
% Add label to reference the table
[ADD_PP_ANALYSIS]
a) **Pattern Introduction**: [Describe how the pattern is introduced]
b) **Pattern Function**: [Explain what the pattern does]
c) **Human Benefit**: [Describe how it helps users]
d) **Reusability**: [Explain how it can be adapted]

%% Expected response. Put the human feeling into the writing. How do I feel when I view the output.
[ADD_HUMAN_FEELING_RESPONSE]
When encountering outputs from this pattern, users typically experience [FEELING_DESCRIPTION]. 
The response [EMOTIONAL_IMPACT] and provides [COGNITIVE_BENEFIT].

%% Re-use: how to derive a PE from PP
[ADD_REUSE_GUIDANCE]
To derive a Prompt Example (PE) from this Prompt Pattern (PP):
1. [REUSE_STEP_1]
2. [REUSE_STEP_2] 
3. [REUSE_STEP_3]

%4 - PP example in this category

\begin{table}[h!]
\fontsize{9pt}{10pt}\selectfont
\centering
\caption{Comparison of Outputs PP}
\label{tab:Comparison_of_Outputs_PP}
\begin{tabular}{|l|}
    \hline
    \textbf{Prompt Pattern} \\ \hline
    \textbf{ID}: 32-2-1\\ 
    \textbf{Category}: CMP\\ 
    \textbf{Name}: Comparison of Outputs\\ 
    \textbf{Media Type}: Text Only, Image2Text, Video2Text\\ 
    \textbf{Description}: The prompt compares outputs by identifying strengths and weaknesses, noting areas of excellence\\ or shortcomings, and providing constructive feedback. Adopting a teacher's role, the AI model offers a balanced\\ comparison, highlighting key differences and similarities to aid in understanding and refining the outputs. \\
    \textbf{Template}: Can you compare the two outputs above as if you were a teacher?\\
    \textbf{Example}: 32-2-1-0\\ 
    \textbf{Related PPs}: \\ 
    \textbf{Reference:} \cite{Bubeck2023SparksGPT-4}\\ \hline
\end{tabular}
\end{table}
            

% Keywords: comparison, contrast, analysis, evaluation, feedback
% Examples: Output comparison with teacher perspective, Strengths and weaknesses analysis, Constructive feedback generation
