% Template for Argument Category under Across Logic
% Use this template to create comprehensive write-ups following the academic structure

\subsection{Argument}
\label{subsec:Argument}
% 3.1 the role of this category under the "across-logic" (meaning of the category)
Refers to a structured process where a claim or viewpoint is presented and defended. This type of prompt enables the AI model to generate a response that not only states a position, but also provides reasoning and evidence to support it.

[ADD_DETAILED_CATEGORY_ROLE_DESCRIPTION]

% 3.2 a. Introduce one PP of the category, b. what the PP did, c. How the PP helps people and d. can be re-used
% Add label to reference the table
[ADD_PP_ANALYSIS]
a) **Pattern Introduction**: [Describe how the pattern is introduced]
b) **Pattern Function**: [Explain what the pattern does]
c) **Human Benefit**: [Describe how it helps users]
d) **Reusability**: [Explain how it can be adapted]

%% Expected response. Put the human feeling into the writing. How do I feel when I view the output.
[ADD_HUMAN_FEELING_RESPONSE]
When encountering outputs from this pattern, users typically experience [FEELING_DESCRIPTION]. 
The response [EMOTIONAL_IMPACT] and provides [COGNITIVE_BENEFIT].

%% Re-use: how to derive a PE from PP
[ADD_REUSE_GUIDANCE]
To derive a Prompt Example (PE) from this Prompt Pattern (PP):
1. [REUSE_STEP_1]
2. [REUSE_STEP_2] 
3. [REUSE_STEP_3]

%4 - PP example in this category

\begin{table}[h!]
\fontsize{9pt}{10pt}\selectfont
\centering
\caption{Debater PP}
\label{tab:Debater_PP}
\begin{tabular}{|l|}
    \hline
    \textbf{Prompt Pattern} \\ \hline
    \textbf{ID}: 11-0-9\\ 
    \textbf{Category}: ARG\\ 
    \textbf{Name}: Debater\\ 
    \textbf{Media Type}: Text\\ 
    \textbf{Description}:  Debater engages the user in a structured debate format. The user is tasked with researching\\ current event topics, presenting balanced arguments for both sides, refuting opposing viewpoints,\\ and drawing evidence-based conclusions. The goal is to enhance the user's understanding and insight\\ into the topic through a comprehensive and persuasive discussion. \\
    \textbf{Template}: I want you to act as a debater. I will provide you with some topics related to current events\\ and your task is to research both sides of the debates, present valid arguments for each side, \\ refute opposing points of view, and draw persuasive conclusions based on evidence. \\Your goal is to help people come away from the discussion with increased knowledge \\and insight into the topic at hand. My first request is "I want an opinion piece about:"\\
    \textbf{Example}: 11-0-9-0\\ 
    \textbf{Related PPs}: 26-0-1, 8-0-0, 26-0-3, 22-0-2, 26-0-0, 22-2-3, 41-2-7, 23-0-0, 40-0-0, 29-0-0\\ 
    \textbf{Reference:} \cite{Akin202450Prompts}\\ \hline
\end{tabular}
\end{table}
            

% Keywords: argument, debate, reasoning, evidence, viewpoint
% Examples: Debater PP for structured debate format, Opinion piece generation, Evidence-based argumentation
