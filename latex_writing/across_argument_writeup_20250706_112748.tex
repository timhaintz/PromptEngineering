% LaTeX Write-up for Argument Category
% Logic: Across - Multiple domains/disciplines integration
% Generated on: 2025-07-06 11:27:48


\subsection{Argument}
\label{subsec:Argument}
% 3.1 the role of this category under the "across-logic" (meaning of the category)
Arguments serve as the foundation for logical reasoning across different domains, enabling the AI to construct coherent positions that bridge multiple areas of knowledge.

% 3.2 a. Introduce one PP of the category, b. what the PP did, c. How the PP helps people and d. can be re-used
% Add label to reference the table

a) **Pattern Introduction**: The Debater pattern establishes a structured framework for presenting and defending viewpoints
b) **Pattern Function**: It guides the AI to research multiple perspectives, present balanced arguments, and draw evidence-based conclusions  
c) **Human Benefit**: Users gain deeper understanding through comprehensive analysis of complex topics from multiple angles
d) **Reusability**: The debater structure can be adapted for any topic requiring multi-perspective analysis


%% Expected response. Put the human feeling into the writing. How do I feel when I view the output.

When encountering outputs from this pattern, users typically experience a sense of intellectual engagement and clarity, as complex topics are broken down into manageable, well-reasoned arguments. 
The response reduces cognitive overwhelm when dealing with controversial or complex subjects and provides structured thinking and evidence-based reasoning.


%% Re-use: how to derive a PE from PP

To derive a Prompt Example (PE) from this Prompt Pattern (PP):
1. Identify a specific topic or issue requiring multi-perspective analysis
2. Adapt the debater template to focus on the chosen topic
3. Customize the evidence requirements and conclusion format for the specific domain


%4 - PP example in this category

\begin{table}[h!]
\fontsize{9pt}{10pt}\selectfont
\centering
\caption{Debater PP}
\label{tab:Debater_PP}
\begin{tabular}{|l|}
    \hline
    \textbf{Prompt Pattern} \\ \hline
    \textbf{ID}: 11-0-9\\ 
    \textbf{Category}: ARG\\ 
    \textbf{Name}: Debater\\ 
    \textbf{Media Type}: Text\\ 
    \textbf{Description}:  Debater engages the user in a structured debate format. The user is tasked with researching\\ current event topics, presenting balanced arguments for both sides, refuting opposing viewpoints,\\ and drawing evidence-based conclusions. The goal is to enhance the user's understanding and insight\\ into the topic through a comprehensive and persuasive discussion. \\
    \textbf{Template}: I want you to act as a debater. I will provide you with some topics related to current events\\ and your task is to research both sides of the debates, present valid arguments for each side, \\ refute opposing points of view, and draw persuasive conclusions based on evidence. \\Your goal is to help people come away from the discussion with increased knowledge \\and insight into the topic at hand. My first request is "I want an opinion piece about:"\\
    \textbf{Example}: 11-0-9-0\\ 
    \textbf{Related PPs}: 26-0-1, 8-0-0, 26-0-3, 22-0-2, 26-0-0, 22-2-3, 41-2-7, 23-0-0, 40-0-0, 29-0-0\\ 
    \textbf{Reference:} \cite{Akin202450Prompts}\\ \hline
\end{tabular}
\end{table}
            
